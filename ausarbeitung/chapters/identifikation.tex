\chapter{Patternidentifikation}
Der nächste Schritt auf dem Weg zur lokalisierung der \QRCodes ist das identifizieren der \emph{Finder Patterns}. Dafür wird die durch \OpenCV bereitgestellte \texttt{findContours} Methode verwendet. Sie basiert auf dem Algorithmus von Suzuki und wird eingesetzt um die Konturen aus dem Binärbild zu bestimmen.
\inputCPP[label={lst:findcontours}][][Aufruf der \texttt{findContours} Methode um die Konturen zu bestimmen.]{code/findContours.cpp}
Listing \ref{lst:findcontours} zeigt den Aufruf um alle Konturen des übergebenen Bildes zu erhalten.

\section{Vorgehensweise des Algorithmus von Suzuki}
Das Binärbild wird als durch iterriert. Für jeden Pixel $p_{i,j}$ werden folgende Bedingungen überprüft:
\begin{description}
	\item[outer border] Falls der Farbton des Vorgängers sich vom aktuell betrachteten Pixel unterscheidet. Beispielsweise für den Pixel $p_{i,j} = 1$ und $p_{i,j-1} = 0$ wäre $p_{i,j}$ ein Startpunkt einer \emph{outer border}. 
	\item[hole border] Falls der Farbton der Vorgänger sich in einem bestimmten Abstand gleich war, aber im aktuellen Pixel $p_{i,j}$ unterscheidet. Sei $x > 1$ ein beliebiger Abstand und es gelte $p_{i,j-x} =\ldots = p_{i,j-1}= 1$ und $p_{i,j} = 0$, dann ist $p_{i,j}$ ein Startpunkt einer \emph{hole border}.
\end{description}
Sind beide Bedingungen erfüllt, ist der Pixel $p_{i,j}$ ein Anfangspunkt einer neuen Kontur. Diese Kontur muss eindeutig identifizierbar sein daher wird sie mit einer \emph{KonturID} versehen. Der große Vorteil dieses Algorithmus ist die abgespeicherte Hierarchie der Konturen. Um dies zu gewährleisten, muss die \emph{parent}-Kontur für die neue Kontur gesetzt werden. Während des Scannens des Bildes wird immer die äußere Kontur zwischengespeichert, diese ist entweder die \emph{parent}-Kontur oder die Kontur die, die neue Kontur und die \emph{parent}-Kontur teilt. Wenn alle Werte gesetzt sind, wird die Kontur durch sukzessives Hinzunehmen von Pixeln erzeugt. Nach jeder Kontur Erzeugung springt der Algorithmus zurück zum Raster Scan. Der Algorithmus terminiert bei Erreichen der rechten unteren Ecke.

Das Orginal Paper beschreibt ausfürlich das Vorgehen anhand Beispielen und Pseudocode.

\section{Filtern der Konturen}
Nachdem die Konturen bestimmt wurden, müssen die Konturen gefiltert werden da abhängig vom Bild können eine variable Anzahl an Konturen
erkannt werden. Beispielsweise werden bei der Ausführung der adaptiven Schwellwertoperation viele kleine Konturen erkannt. In folgenden Fällen werden Konturen ignoriert, da sie keine Aussagekraft besitzen.
\begin{enumerate}
	\item Die Kontur ist zu klein oder zu groß.
	\item Die Kontur besitzt kein \emph{parent}-Kontur.
	\item Konturen die Nachbarkonturen besitzen.
	\item Konturen die eine \emph{child}-Kontur besitzen.
\end{enumerate}
Für die äußeren Konturen also die \emph{parent}-Konturen müssen die Bedingungen 1.-3. natürlich auch gelten. Zusätzlich muss für die äußerste Kontur eine Trapezoide Form gelten.


