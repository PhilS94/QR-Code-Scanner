\chapter{Extrahierung}
\section{Pattern Kanten approximieren}

Kantengeraden bestimmen (Besserer Titel)

Nachdem mehrere Kandidaten für Finder Patterns gefunden wurden, wird als nächstes die äußerste Kontur dieser Kandidaten betrachtet. Zuvor haben wir alle Konturen ohne Approximation erstellt, weswegen jede Kontur jetzt eine zusammenhängende Kette von Punkten darstellt. Für jeden Kandidat soll mit der Hilfe dieser Punkte die Vier Kanten der Finder Patterns bestimmt werden. Dazu teilen wir die Kontur in Vier Segmente auf und benutzen diese im späteren Verlauf um die Kantengeraden zu approximieren. Um die Schnittpunkte zu finden, an welchen die Kontur geteilt werden muss, verwenden wir die zuvor berechneten vier Punkte, welche genutzt wurden um festzustellen ob es sich bei dir Kontur um ein Trapezoid handelt. (Erklären auf welche Weise diese Vier Punkte approximiert wurden) Diese Punkte liegen allerdings nicht immer exakt auf den Ecken der Patterns und außerdem kann es durch schlechte Bildqualität zu abgerundeten Ecken im Bild kommen. Deswegen werden zusätzlich nachdem die Kontur zerteilt wurde jeweils 10\% aller Punkte an den Enden jedes Segments verworfen, sodass ein Segment 80\% aller Punkte nachdem zerteilen behält. Diese Segmente werden dann verwendet um mit Hilfe der fitLine Methode (Genauere Funktionsweise der fitLine Methode erläutern. Feststellen das der berechnete Stützvektor sich immer in der Mitte des Segments befindet. Sagen des es sich um Stütz und Richtungsvektor handelt.) die Kantengeraden für jeden Finder Pattern Kandidaten zu bestimmen.

Das Ergebnis ist, dass für jeden Kandidaten die äußerste Kontur, die vier Segmente welche die Kanten beschreiben sollen und die approximierten Kantengeraden bekannt sind.



Finder Pattern Kombinationen evaluieren

Der nächste Schritt besteht darin alle möglichen Kombinationen von drei Finder Patterns zu prüfen und festzustellen ob diese zur Extrahierung eines gültigen QRCodes führen. Dies führt zu (n über k) (richtig aufschreiben) Kombinationen von Finder Patterns. Für Bilder in denen fälschlicher Weise viele Finder Patterns erkannt werden oder viele Patterns vorhanden sind, kann dies schnell zu großem Rechenaufwand führen. Deswegen gibt es im folgenden mehrere Stellen an denen Kombinationen verworfen werden, sobald deutlich wird das diese zu keinem korrekten Ergebnis führen können.

Es werden also Drei Finder Patterns ausgesucht welche zunächst zufällig in einem Array liegen. Damit es von hier aus möglich ist den QRCode korrekt zu extrahieren muss bestimmt werden welches dieser Patterns welche Position im QRCode einnimmt (Oben Links/Oben Rechts/Unten Links). Um dies korrekt zu tun ordnen wir die Patterns zuerst so an, wie sie im Uhrzeigersinn im Bild zu finden sind. Dazu verwenden wir die Gaußsche Trapezformel. (Formel aufschreiben und Flächen Berechnung erklären). Sind die Patterns nun in korrekter Reihenfolge, bestimmen wir als nächstes das obere linke Pattern.

Zurzeit haben wir durch die drei Patterns insgesamt 12 approximierte Kanten. Da die Finder Patterns in einem korrekten QRCode allerdings so ausgerichtet sind, dass das obere linke Pattern ausschließlich Kanten hat die mit einem der anderen beiden Patterns identisch sind, haben wir nur 8 einzigartige Kanten. Stellt man also fest welche der 12 Geraden die gleiche Kantengerade im QRCode beschreiben kann festgestellt werden, welches Pattern das obere linke Pattern ist.

Um dies effizient zu tun, nutzen wir aus das die Stützvektoren der Kantengeraden stets im  Mittelpunkt der Segmente und somit auch ungefähr im Mittelpunkt der echten Kanten des Patterns liegen. Ist dies nämlich der Fall kann man ein Ähnlichkeitsmaß für zwei Geraden definieren über die Summe der Entfernungen der Stützvektoren einer Gerade zur anderen Gerade. (Bild für Verständnis) Ein kleinerer Wert bedeutet in diesem Fall dann das sich zwei Geraden besonders ähnlich sind. In Abbildung (Bla) ist dabei gut zu erkennen, dass wenn es sich um einen QRCode handelt, welcher mit ausreichend guter Qualität für eine korrekte Kanten approximation abgebildet ist, das sich dieses Maß sehr gut eignet um festzustellen welche der insgesamt 12 Kantengeraden die gleiche Kante approximieren. Für das obere linke Pattern gilt dann in diesem Fall, dass es das Pattern mit den insgesamt vier kleinsten Ähnlichkeitswerten ist.

Da zuvor die Patterns im Uhrzeigersinn angeordnet wurden, können wir durch das Rotieren im Uhrzeigersinn nicht nur das obere linke Pattern an die erste Stelle des Arrays bringen, sondern wissen auch direkt das sich an der zweiten Stelle das obere rechte Pattern und an der dritten Stelle das untere linke Pattern befindet. Mit der Ausnahme natürlich für Spiegel verkehrte QRCodes von denen es hier nicht das Ziel ist diese auch zu erkennen.

Da wir bereits berechnet haben, welche der Kantengerade die selbe Kante approximieren, können wir diese Informationen benutzen um eine qualitativ deutlich bessere Approximation für alle Kanten des oberen linken Patterns zu berechnen. Wir nehmen dazu die zugrunde liegenden Segmente von den gepaarten Kanten und führen auf der Menge dieser Punkte erneut eine Approximation der Kantengeraden durch. Indem wir dabei feststellen ob wir mit einer Kante aus dem oberen rechten oder dem unteren linken Pattern vereinigen, können wir feststellen ob die neue Kante im QRCode Koordinatensystem eine horizontale oder vertikale Kante beschreibt. Die nicht verwendeten Kanten am Ende des Prozesses sind danach ebenfalls trivial zuzuordnen. (Falls an dieser Stelle nicht 4-2-2 gemerged wurde, abbrechen.)

An dieser Stelle kennen wir dann vier horizontale und vier vertikale Kantengeraden des QRCodes die durch die Finder Patterns beschrieben werden.

(Sagen dass wir das machen weil die anderen Mehtoden nicht zu 100% perspektivisch sicher, bzw. Berechnungstechnisch weniger aufwändig sind.) Um festzustellen welche von diesen Kanten die Äußeren Kanten sind, sortieren wir jeweiles vier Kanten einer Richtung entlang einer beliebigen Kante der anderen Richtung. Dazu berechnen wir die Schnittpunkte der vier Geraden mit der gewählten Achse, rotieren diese, sodass sich alle Schnittpunkte auf der X-Achse bewegen (also rotation ins Achsensystem) und sortieren dann schlicht nach kleinstem zu größten x wert.

Nachdem dies